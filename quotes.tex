
\begin{enumerate}
\item ** [On requesting access to data] "If the mechanism for accessing data is 'email someone and wait for them to respond', then if you don't ideally have both professor and OBE involved in your name, you're going to struggle."
\item [On badly curated covid data] "essentially, you had a treasure trove of information that was not at all mapped to each other, that a lot could have been done with, which was being heavily access managed and not at all curated."
\item ** [on access managing non-private data] "When you access manage to try and protect individuals' privacy, you're doing so for a specific reason [...] Things do slow down, but there's a reason you're doing it. This just felt like wasted effort, where you're access managing something you really shouldn't be bothering with."
\item ** [informing policy with data that warns against being used for policy] "Guides on how to do that from the government come with a big warning sheet saying you should under no circumstances use this for any software that informs policy, which is not very helpful... because we have to."
\item [data that can't be re-distributed isn't useful] "The GISAID database which is containing the SARS-CoV-2 sequences is blocked  off, and it's kind of not worth the hassle for use to that data, even though it might be informative, because if we use it, we then can't share our data, which is a derivative."
\item ** [Working on two different data sources that don't have overlapping sharing permissions] "Having access to highly protected things, like healthcare data, for half of my work is bizarre. because for example, we need to know what proportion of people who enter the hospital over the age of seventy dies? what is that rate? and I know it, because I generated it for one of my models, but I can't use it for the other model".
\item  [Data changes shape all the time] "Open but annoying sometimes to deal with, but that's not because they're un-open, it's because they change their geographies every so often and they don't match, and things like that".
\item  [on GISAID] "that makes no sense to me, because [...] we need to make our database public and accessible, so if we are to use and of these restricted datasets, then we would have to restrict parts and essentially engineer a two-tier system into what we're doing, and that's just not worth the hassle."
\item [on open virus sequences] "Everything should just be public. I understand that there's patient confidentiality, [...] but I don't understand how virus sequence alone - especially assembled virus, these are not raw reads - is in any way patient identifying."
\item [on using GISAID for analyses] "To me, this is the biggest [...] it doesn't make sense to use that data because of the strings attached."
\item "The fact that we're in a covid-19 pandemic and people are not releasing coronavirus sequences that will help with evolutionary analysis - this doesn't really make much sense to me"
\item [on rapid datasharing by JCVI] "and they just dropped it [deposited publicly], because that's I think the morally right thing to do. You don't know which of these datasets will help with therapeutic development or not, so how can you make the judgement that this dataset is not worth helping the effort?"
\item [On animals vs humans] "When there's an animal disaster, the data seem to flow immediately, for better or for worse it becomes much less cautious [...] but in the human case, this hasn't happened. Probably that's good, because peoples' private data is protected. Certainly for hospital level health records of individual progress through the disease, I 100 percent understand why those have been strongly protected, but things like what percentages of tests have been positive in different cities - how many deaths in community vs hospital, or deaths by age category. All of those things - still not available publicly, and I don't think you can make a good argument that those are private." - adds a caveat that small populations this could still need to stay private.
\item [on datasharing attitudes] "They don't want to make themselves vulnerable by sharing their sequences, but if everyone does this, the whole field is held back."
\item [on SITREP format] "We had to find a way to synthesise about 120 excel documents, where within each excel document there was several sheets [...] and it was only about a week of working with these that we realised there were several hidden sheets, and if you right clicked and unhid those sheets you suddenly had access to the raw data."
\item  [on datasharing attitudes] "I guess this was okay in the 90s when sharing sequences was hard"
\item "It's just policy thing - we are third party. We are not the original data collectors - If you're a bioinformatician and you analyse someone else's data, that has less merit than if I sequence the data myself and analyse the data myself."
\item [on anti-datasharing attitudes in virology] "It is kind of a shame that we're holding each other back [...] It really requires leadership from someone respected in the field to say okay, 'we have to do things differently'."
\item [Re wishing better metadata] "Right now it's really common that I'll get a file that says region, number of schools, something like that. And the way I find what the regions are is that I pick one out and I Google it to see if it appears in my neighborhood table, my council table, or some other geography table because they're all codes with the exact same format."
\item [On finding SITREP by accident] "I found it somewhat amusing that it was only on an off-chance that I was perusing through these documents [...] then we stumbled across SITREP, an annual daily submission as to their current bed status. " - different types of  beds, e.g. intensive care, critical care, general, mental health, etc.
\item  [re: SITREP: Access managing as a habit] Equally, we were pushing forward saying we want to get this out there, because it shouldn't be something that is being managed in the way that it is. When you access manage to try and protect individuals' privacy, you're doing so for a specific reason, and it's okay to play off the harms of that to the research process. Things do slow down, but there's a reason you're doing it. This just felt like wasted effort, where you're access managing something you really shouldn't be bothering with.
\item [on data use for evidence-based policies in government] "It probably would have been a problem before the pandemic. We've worked with local governments over the past few years, and there are continuous issues with how to get them using data, and how to build the internal uses-case of why data matters, and why they should look at the data they already have access to that may be siloed in different teams. There's a foundational problem that was definitely exacerbated by the stress and the business of the pandemic"
\item [on industry data culture] "Sharing more data - it kind of needs all of them to do it, or none of them to do it. There's a little bit of game theory involved - they don't want to lose out on their competitive advantage."
\item [on sharing data across an industry (such as safety data) for the benefit of all] "There needs to be a lot more work on building that shared vision - I think most private sector companies just had not reckoned with the fact that there is a wider societal impact, and there may be financial benefits, but really we're talking about wider societal and sector wide benefit, which is a really hard thing to drive into people who just look at profit as their number one motivation."
\item [on Strava metro] Strava make a real big song and dance about it, but their restrictions on the re-use of that data are outrageous. They require - you can't use it for COVID stuff, which is just bizarre. If you want to use any of it publicly, you need to go back to strava and give them a veto over what you do with that data. [...] even to publish data they've shared with you, when you're not providing any kind of subjective lens on top of that - you need their permission, and they can veto you publishing it."
\item ** [Strava metro continued] "I spoke to one person from local government who just went 'we're just not going to use it then' - there's just no value to it if you're going to provide that amount of restriction, and then it starts to look like a concept of open washing, where we're saying 'look at all the good stuff we're doing!", then dedicate the bare minimum to doing something open, but not fundamentally change anything."
\end{enumerate}